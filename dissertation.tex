\documentclass[10pt, pdf,xcolor=pdftex,dvipsnames,table]{beamer}
\usepackage[brazil]{babel}
\usepackage[utf8]{inputenc}
\usepackage[T1]{fontenc}
\usepackage{pgfpages}
\usepackage{times}
\usepackage{amsmath,amssymb}
\usepackage{graphicx}
\usepackage{color}
\usepackage{hyperref}
\usepackage{pxfonts,txfonts}
\usepackage{url}
\usefonttheme{structurebold}
\usepackage{hyphenat}
\usepackage{multicol}
\usepackage{fp}
\usepackage{fancyvrb}
\usepackage{xcolor}
\usepackage[caption=false]{subfig}
\usepackage{pgfplots}
\usepackage{pgfplotstable}
\usepackage{tikz}
\usetikzlibrary{patterns}
\input{preamble/colordefines.tex}


\newcommand{\x}{\textbf{\textcolor{Green}{$\surd$}}}
\newcommand{\xx}{\textbf{\textcolor{Blue}{$\odot$}}}
\newcommand{\xxx}{\textbf{\textcolor{Red}{$\times$}}}

\newcommand{\tf}{\cellcolor{green!65}}
\newcommand{\tp}{\cellcolor{blue!65}}
\newcommand{\tn}{\cellcolor{yellow!65}}


\usetheme{Amsterdam}
\setbeamertemplate{navigation symbols}{}

\pgfdeclareimage[height=1.5cm]{logo}{images/lups_oficial.png}
\logo{
	\hspace{5cm}
	\pgfuseimage{logo}
}

\setbeameroption{hide notes}

%==================================================================================

%EVENTO
\renewcommand{\evento}{Defesa de Mestrado}

% TITULO DA APRESENTACAO
\title{LTMS - Lups Transactional Memory Scheduler: Um escalonador NUMA-Aware para STM}

%Autor
\author{\textbf{Michael Alexandre Costa}\\
\and Prof. Dr. André Rauber Du Bois (Orientador) \\
}

%%%%%%%%%%%%%%%%%%%%%%%%%%%%%%%%%%%%%%%%
% Instituição
%%%%%%%%%%%%%%%%%%%%%%%%%%%%%%%%%%%%%%%%

\institute{Mestrado em Computação \\ Centro de Desenvolvimento Tecnológico \\ Universidade Federal de Pelotas \\
\url{macosta@inf.ufpel.edu.br} 
}

\date{\today}

\begin{document}

% % % % % % % % % % % % % % % % % % % % % %
\frame{\titlepage}
\pgfdeclareimage[height=0.7cm]{logo}{images/lups_timbre.png}
\logo{
	\pgfuseimage{logo}
}


% % % % % % % % % % % % % % % % % % % % % %
\frame{\tableofcontents}

%%%%%%%%%%%%%%%%%%%%%%%%%%%%%%%%%%%%%%%%%%%%%
% Conteúdo da Apresentação
%%%%%%%%%%%%%%%%%%%%%%%%%%%%%%%%%%%%%%%%%%%%%

\section{Introdução}

\begin{frame} \frametitle{Introdução}
    \begin{block}{Motivação}
        \begin{itemize}
        	\item Programação Paralela;
        	\item Memórias Transacionais;
        	\item Escalonadores de Transações; e
        	\item Arquiteturas NUMA.
        \end{itemize}
    \end{block}
\end{frame}

\begin{frame} \frametitle{Introdução}
    \begin{block}{Objetivos}
        \begin{itemize}
        	\item Investigar escalonadores de Memórias Transacionais em NUMA.
        \end{itemize}
    \end{block}
    \begin{block}{Contribuições}
        \begin{itemize}
        	\item Projeto de um escalonador de STM intitulado LTMS;
        	\item Prototipação do escalonador LTMS; e
			\item Análise de desempenho do LTMS comparado a TinySTM.
        \end{itemize}
    \end{block}
\end{frame}

\begin{frame} \frametitle{Introdução}
    \begin{block}{Características}
        \begin{itemize}
        	\item Mecanismo para leitura da arquitetura e criação de filas;
        	\item Duas diferentes heurísticas de distribuição inicial de threads;
        	\item Mecanismo que em tempo de execução coleta informações sobre as threads;
        	\item Mecanismo de migração de threads entre as filas de execução; e
        	\item Duas heurísticas de migração.            
        \end{itemize}
    \end{block}
\end{frame}
% • Um mecanismo para leitura da arquitetura e criação de filas com conhecimento sobre os nodos NUMA utilizados, onde cada fila utilizará um único núcleo disponível da arquitetura;
% • Duas diferentes heurísticas de distribuição inicial de threads que podem ser aplicadas para compreender o impacto da distribuição inicial de threads em aplicações paralelas;
% • Um mecanismo que em tempo de execução coleta informações sobre as threads e armazena os endereços de memória mais utilizados por cada thread, para então mensurar com base nos nodos NUMA utilizados os diferentes custo de acesso à memória;
% • Um mecanismo de migração de threads entre as filas de execução, que permite a serialização de threads conflitantes e não reduz o paralelismo da aplicação; e
% • Duas heurísticas de migração adicionadas ao escalonador para estudar e compreender o impacto que a redução da latência e o alto índice de conflitos possuem sobre aplicações de STM.

\section{Conceitos Abordados}
\begin{frame} \frametitle{Conceitos abordados no trabalho}
    \begin{block}{Principais Conceitos}
        \begin{itemize}
        	\item Memórias Transacionais;
        	\item Escalonadores de transações; e
        	\item Arquiteturas.
        \end{itemize}
    \end{block}
\end{frame}

% \section{Memórias Transacionais}
\begin{frame} \frametitle{Memórias Transacionais}
    \begin{block}{Características}
        \begin{itemize}
        	\item Fornece abstração de código; e
        	\item Ausência de deadlocks.
        \end{itemize}
    \end{block}
    
    \begin{block}{Transações}
        \begin{itemize}
        	\item Atomicidade;
        	\item Consistência; e
        	\item Isolamento.
        \end{itemize}
    \end{block}
\end{frame}

\begin{frame} \frametitle{Memórias Transacionais}
    \begin{alertblock}{Problemas}
        \begin{itemize}
        	\item Somente reinicia a transação conflitante;
        	\item Não evita que conflitos futuros aconteçam; e
        	\item Em ambientes de alta contenção, tende a perder desempenho.
        \end{itemize}
    \end{alertblock}
\end{frame}

% \section{Escalonadores}
\begin{frame} \frametitle{Escalonadores}
\begin{block}{Escalonadores de Transações}
\begin{itemize}
	\item Buscam reduzir os números de conflitos;
	\item Utilizam diferentes Heurísticas de escalonamento; e
	\item Serializa as transações conflitantes.
\end{itemize}
\end{block}
\end{frame}

\begin{frame} \frametitle{Escalonadores}
    \begin{block}{Trabalhos Estudados}
        \begin{itemize}
            \item ATS;
            \item CAR-STM;
            \item Shrink;
            \item LUTS;
            \item ProVIT; e
            \item STMap.
        \end{itemize}
    \end{block}
\end{frame}

% \begin{frame} \frametitle{Escalonadores}
% \begin{block}{Classificação das técnicas}
% \begin{itemize}
% 	\item Baseado em Heurística:
% 	\begin{itemize}
% 	    \item Feedback;
% 	    \item Predição;
% 	    \item Reativo; e
% 	    \item Heurística Mista.
% 	\end{itemize}
% 	\item Baseado em Modelo:
% 	\begin{itemize}
% 	    \item Aprendizado de Máquina;
% 	    \item Modelo Analítico; e
% 	    \item Modelo Misto.
% 	\end{itemize}
% \end{itemize}
% \end{block}
% \end{frame}


\begin{frame} \frametitle{Escalonadores}
\begin{table}[]
    \footnotesize
    \caption{Comparativo entre os escalonadores apresentados}
    \label{tab:compare_ltms}
    \resizebox{\columnwidth}{!}{
    \begin{tabular}{l|l|l|l|l|l|l|l}
    \hline
     Escalonadores                     & LTMS    & STMap     & ATS      & Shrink   & LUTS  & ProVIT  & CAR-STM \\ \hline
     Distribuição inicial de threads   & Sim     & Não       & Não      & Não      & Sim   & Não     & Não \\
     Coleta de dados por threads       & Sim     & Sim       & Não      & Sim      & Não   & Não     & Não \\
     Migração entre filas              & Sim     & Não       & Não      & Não      & Não   & Não     & Sim \\
     Avalia a arquitetura              & Sim     & Sim       & Não      & Não      & Não   & Não     & Não \\
     NUMA                              & Sim     & Sim       & Não      & Não      & Não   & Não     & Não \\
    %  Técnica de escalonamento          & Reativo & Predição  & Feedback & Predição & Mista & Mista   & Reativo  \\
    \hline
    \end{tabular}
    }
   \end{table}
\end{frame}

% \section{Arquiteturas}
\begin{frame} \frametitle{Arquiteturas}
\begin{block}{UMA}
\begin{itemize}
	\item Uniform Memory access;
	\item Possui um único barramento de acesso à memória; e
	\item Único custo de acesso à memória.
\end{itemize}
\end{block}
\begin{block}{NUMA}
\begin{itemize}
	\item Non-uniform Memory access;
	\item Possui mais de um barramento de acesso à memória; e
	\item O custo de acesso à memória é diferente conforme o núcleo utilizado.
\end{itemize}
\end{block}
\end{frame}

\section{LTMS}
\begin{frame} \frametitle{LTMS}
    \begin{block}{Estágios}
        \begin{itemize}
        	\item Inicialização do sistema;
        	\item Coleta de dados em tempo de execução; e
        	\item Migração de Threads.
        \end{itemize}
    \end{block}
\end{frame}

\begin{frame} \frametitle{LTMS}
    \begin{figure}[!h]
        \centering
        \includegraphics[scale=0.2]{images/LTMS1}
        \caption{Fluxograma do LTMS}
        \label{fig:abusy}
    \end{figure}
\end{frame}

\begin{frame} \frametitle{LTMS - Estágio 1}
    \begin{block}{Inicialização do sistema}
        \begin{itemize}
        	\item Criação de filas; e
        	\item Distribuição das threads.
        \end{itemize}
    \end{block}

    \begin{block}{Heurísticas de Distribuição}
        \begin{itemize}
        	\item Sequential; e
        	\item Chunks.
        \end{itemize}
    \end{block}
\end{frame}

\begin{frame} \frametitle{LTMS - Heurísticas}
    \begin{figure}[!h]
        \includegraphics[scale=0.4]{images/Queue_one}
        \caption{Heurística Sequential}
        \label{fig:abusy}
    \end{figure}
\end{frame}

\begin{frame} \frametitle{LTMS - Heurísticas}
    \begin{figure}[!h]
        \includegraphics[scale=0.4]{images/Queue_chunks}
        \caption{Heurística Chunks}
        \label{fig:abusy}
    \end{figure}
\end{frame}

\begin{frame} \frametitle{LTMS - Estágio 2}
    \begin{block}{Coleta de dados em tempo de execução}
        \begin{itemize}
        	\item Aborts e Commits;
        	\item Matriz de Comunicação; e 
        	\item Matriz de Endereços.
        \end{itemize}
    \end{block}
\end{frame}

\begin{frame} \frametitle{LTMS - Matrizes}
    \begin{block}{Matriz de Comunicação}
        \begin{itemize}
        	\item Eventos de comunicação são acessos em comum à memória;
        	\item Quantidade de eventos de comunicação entre pares de threads; e
        	\item É coletado 1 evento de comunicação a cada 100 acessos.
        \end{itemize}
    \end{block}
    % A matriz de comunicação fornece insumos sobre a quantidade eventos de comunicação entre dois threads, onde cada posição da matriz representa a quantidade de comunicação entre pares de threads.
\end{frame}

\begin{frame} \frametitle{LTMS - Matrizes}
    \begin{block}{Matriz de Endereços}
        \begin{itemize}
        	\item Endereços em comum mais acessados entre os pares de threads;
        	\item Utiliza uma Tabela Hash;
        	\item Chave: Endereços de memória; e 
            \item Valor: Quantidade de acessos recebidos.
        \end{itemize}
        % possui para cada posição uma tabela hash que contem uma estrutura de chave e valor. Esta estrutura utiliza como chave o endereço de memória acessado e como valor, a quantidade de acessos que este endereço recebeu.Quando duas threds acessam o mesmo endereço de memória, um evento é disparado, este evento busca na tabela hash a chave com endereço que foi acessado e incrementa o valor de acessos que este endereço recebeu.
    \end{block}
\end{frame}

\begin{frame} \frametitle{LTMS - Estágio 3}
    \begin{block}{Migração de Threads}
        \begin{itemize}
        	\item Quando ocorre um abort;
        	\item Identificar a melhor fila; e
            \item Heurísticas de migração.
        \end{itemize}
    \end{block}
\end{frame}

\begin{frame} \frametitle{LTMS - Filas e Threads}
    \begin{block}{Escolha das filas}
        \begin{itemize}
        	% \item Identifica os pares de threads;
        	\item Identifica a fila que possui a thread com mais acessos em comum;
        	\item Utiliza a matriz de comunicação; e
        	\item Busca uma melhor coerência de cache.
        \end{itemize}
    \end{block}
    % A etapa de identificação das threads conflitantes busca entender a aplicação e a arquitetura para definir para qual fila a thread que gerou o abort deve ser migrada.
\end{frame}

\begin{frame} \frametitle{LTMS - Heurísticas}
    \begin{block}{Threshold}
        \begin{itemize}
        	\item Avalia o nível de contenção (Abort/Commit);
        	\item Limiar alto permite maior contenção e gera menos migrações;
        	\item Limiar baixo permite menor contenção e gera mais migrações; e
        	\item Limiar de 0.8 (80\% de contenção).
        \end{itemize}
    \end{block}
\end{frame}

\begin{frame} \frametitle{LTMS - Heurísticas}
    \begin{block}{Latency}
        \begin{itemize}
        	\item Avalia a latência de acesso à memória;
        	\item Matriz de endereços;
        	\item Nodos NUMA; e
        	\item Bancos de memória.
        \end{itemize}
    \end{block}
\end{frame}

\section{Experimentos}
\begin{frame} \frametitle{Experimentos}
    \begin{block}{Aplicação}
        \begin{itemize}
            \item TinySTM 1.0.5; e
            \item STAMP 0.9.10.
        \end{itemize}
    \end{block}
 
    \begin{block}{Arquitetura}
        \begin{itemize}
        	\item Intel Xeon E5-4650;
            \item 96 núcleos e 192 threads;
            \item 468Gb de memória RAM.
        \end{itemize}
    \end{block}
\end{frame}

\begin{frame} \frametitle{Experimentos}
    \begin{block}{Testes}
        \begin{itemize}
        	\item Cenários de threads: 
            \begin{itemize}
                \item 1, 2, 4, 8, 16, 32, 64, 128, 256, e 512;
            \end{itemize}
            \item Heurísticas de Distribuição-Migração:
            \begin{itemize}
                \item Sequential-Threshold;
                \item Chunks-Threshold;
                \item Sequential-Latency;
                \item Chunks-Latency;
            \end{itemize}
            \item TinySTM; e
            \item Baterias de 30 execuções.
        \end{itemize}
    \end{block}
\end{frame}

\section{Resultados}
\begin{frame} \frametitle{Resultados}
    \begin{block}{Benchmarks}
        \begin{itemize}
        	\item Bayes;
        	\item Intruder;
        	\item Kmeans;
        	\item Labyrinth;
            \item Vacation; e
            \item Yada.
        \end{itemize}
    \end{block}
\end{frame}

\begin{frame} \frametitle{Intruder}
    \begin{figure}[!ht]
    \centering
    \includegraphics[scale=0.3]{images/legend}
    
    \subfloat[Tempo de execução]{
        \label{Intruder}
        \begin{tikzpicture}[scale=0.35, baseline]
        \begin{axis}[
            width=1.5 \linewidth,
            height=1 \linewidth,
            %media de tempo intruder
            ybar=2.5pt,
            % enlargelimits=0.10,
            % legend style={at={(0.45,1.1)}, anchor=south, legend columns=0, nodes={scale=2}},
            ylabel=Tempo (s),
            xlabel=Threads,
            symbolic x coords={1, 2, 4, 8, 16, 32, 64, 128, 256, 512},
            xtick=data,
            ymin=0,
            ymax=400,
            bar width=5pt,
            % nodes near coords,
            nodes near coords align={vertical},
        ]
        \addplot+[error bars,y dir=both, y explicit] coordinates {
            (1,22.49)+-(1,0.11) (2,47.47)+-(2,0.69) (4,36.62)+-(4,0.59) (8,39.90)+-(8,0.89) (16,40.47)+-(16,0.35) (32,40.54)+-(32,3.55) (64,42.59)+-(64,0.29) (128,47.43)+-(128,1.97) (256,122.18)+-(256,4.64) (512,356.05)+-(512,5.11) 
        };
        \addplot+[error bars,y dir=both, y explicit] coordinates {
            (1,27.06)+-(1,0.18) (2,16.76)+-(2,0.13) (4,11.43)+-(4,0.10) (8,8.18)+-(8,0.13) (16,7.20)+-(16,0.06) (32,7.31)+-(32,0.14) (64,8.25)+-(64,0.81) (128,8.00)+-(128,0.37) (256,10.82)+-(256,3.29) (512,11.20)+-(512,0.86)
        };
        \addplot+[error bars,y dir=both, y explicit] coordinates {
            (1,26.90)+-(1,0.17) (2,16.89)+-(2,0.13) (4,11.50)+-(4,0.13) (8,8.10)+-(8,0.10) (16,7.21)+-(16,0.06) (32,7.28)+-(32,0.07) (64,7.61)+-(64,0.63) (128,9.11)+-(128,0.98) (256,10.40)+-(256,2.68) (512,11.59)+-(512,0.72)
        };
        \addplot+[error bars,y dir=both, y explicit] coordinates {
            (1,27.73)+-(1,0.05) (2,16.65)+-(2,0.16) (4,10.50)+-(4,1.10) (8,6.64)+-(8,0.23) (16,24.42)+-(16,2.56) (32,29.00)+-(32,6.41) (64,28.59)+-(64,10.33) (128,25.01)+-(128,0.29) (256,26.19)+-(256,1.25) (512,27.87)+-(512,2.82)
        };
        \addplot+[error bars,y dir=both, y explicit] coordinates {
            (1,27.96)+-(1,0.25) (2,28.12)+-(2,0.28) (4,28.11)+-(4,0.14) (8,28.17)+-(8,0.07) (16,23.01)+-(16,1.76) (32,36.20)+-(32,1.18) (64,30.24)+-(64,4.39) (128,24.97)+-(128,1.10) (256,26.99)+-(256,2.13) (512,27.18)+-(512,2.84)
        };
        % \legend {Tiny, Latency-Sequential, Latency-Chunks, Threshold-Sequential, Threshold-Chunks}
        \end{axis}
        \end{tikzpicture}
    }
    \subfloat[Aborts]{
        \label{abortIntruder}
        \begin{tikzpicture}[scale=0.35, baseline]
        \begin{axis}[
            ymode=log,
            width=1.5 \linewidth,
            height=1 \linewidth,
            %media de tempo intruder
            ybar=2.5pt,
            %enlargelimits=0.10,
            % legend style={at={(0.45,1.1)}, anchor=south, legend columns=-1},
            ylabel=Aborts (log),
            xlabel=Threads,
            symbolic x coords={1, 2, 4, 8, 16, 32, 64, 128, 256, 512},
            xtick=data,
            ymin=0,
            ymax=490000000000,
            bar width=5pt,
            % nodes near coords,
            nodes near coords align={vertical},
        ]
        \addplot+[error bars,y dir=both, y explicit] coordinates {
            (1,0.0)+-(1,0.0) (2,38000260.8)+-(2,2949711.678076852) (4,166325696.4)+-(4,18557044.648063533) (8,529197870.6)+-(8,22132676.255908735) (16,924927047.6)+-(16,24358107.18760883) (32,1183579117.4)+-(32,270222169.9344623) (64,1831169740.8)+-(64,62890886.257802844) (128,2763435636.6)+-(128,1003746150.3805901) (256,138779112147.6)+-(256,10430244205.851603) (512,431443096033.6)+-(512,28705521954.994297) 
        };
        \addplot+[error bars,y dir=both, y explicit] coordinates {
            (1,0.0)+-(1,0.0) (2,724967.4)+-(2,38086.07014959669) (4,2505616.2)+-(4,79429.60770745378) (8,6012716.4)+-(8,169677.49348290125) (16,9639253.0)+-(16,142573.7883104745) (32,10568731.0)+-(32,292443.9680909832) (64,12171771.4)+-(64,1784435.4526392485) (128,19342142.8)+-(128,3669581.5184994815) (256,32980054.25)+-(256,9356623.644361822) (512,34119006.5)+-(512,8983159.5)
        };
        \addplot+[error bars,y dir=both, y explicit] coordinates {
             (1,0.0)+-(1,0.0) (2,745630.2)+-(2,40338.85590792084) (4,2559770.6)+-(4,81679.883166175) (8,6127472.777777778)+-(8,267489.386445634) (16,9889825.3)+-(16,755124.9867329315) (32,10236117.3)+-(32,288410.38292407227) (64,13078693.1)+-(64,982436.3361754746) (128,19063545.666666668)+-(128,2925827.5185046173) (256,33704832.6)+-(256,6772680.448815952) (512,34183921.0)+-(512,8274893.22) 
        };
        \addplot+[error bars,y dir=both, y explicit] coordinates {
            (1,0.0)+-(1,0.0) (2,3808172.2)+-(2,327085.8815720422) (4,9023407.8)+-(4,1123234.4180183227) (8,16872564.2)+-(8,1114383.5354166715) (16,17004377.2)+-(16,2666034.739783216) (32,19648666.0)+-(32,1287522.1930899676) (64,22307089.8)+-(64,4029768.617032169) (128,27162139.0)+-(128,1504557.0) (256,38132893.15)+-(256,4958389.304) (512,37984673.0)+-(512,5282793.33)
        };
        \addplot+[error bars,y dir=both, y explicit] coordinates {
            (1,0.0)+-(1,0.0) (2,3578932.78)+-(2,176854.86) (4,9738434.34)+-(4,1124342.85) (8,17928374.6)+-(8,1573847.9) (16,18301736.6)+-(16,2468736.987185277) (32,22436982.0)+-(32,1106602.0) (64,23420167.2)+-(64,3281430.727250502) (128,23366253.0)+-(128,1919489.0) (256,38736187.88)+-(256,5748878.34) (512,39763372.723)+-(512,3174827.12)
        };
        % \legend {Tiny, Latency-Sequential, Latency-Chunks, Threshold-Sequential, Threshold-Chunks}
        \end{axis}
        \end{tikzpicture}
    }
\end{figure}

\end{frame}

\begin{frame} \frametitle{Kmeans}
    \begin{figure}[!ht]
    \centering
    \includegraphics[scale=0.3]{images/legend}
    
    \subfloat[Tempo de execução]{
        \label{Kmeans}
        \begin{tikzpicture}[scale=0.35, baseline]
        \begin{axis}[
            width=1.5 \linewidth,
            height=1 \linewidth,
            %media de tempo intruder
            ybar=2.5pt,
            %enlargelimits=0.10,
            % legend style={at={(0.5,-0.15)}, anchor=north, legend columns=-1},
            ylabel=Tempo (s),
            xlabel=Threads,
            symbolic x coords={1, 2, 4, 8, 16, 32, 64, 128, 256, 512},
            xtick=data,
            ymin=0,
            ymax=150,
            bar width=5pt,
            % nodes near coords,
            nodes near coords align={vertical},
        ]
        \addplot+[error bars,y dir=both, y explicit] coordinates {
            (1,13.32)+-(1,0.05) (2,18.64)+-(2,1.86) (4,15.33)+-(4,2.68) (8,14.72)+-(8,2.44) (16,12.06)+-(16,0.95) (32,18.24)+-(32,5.19) (64,20.20)+-(64,7.21) (128,32.16)+-(128,15.30) (256,40.54)+-(256,4.28) (512,81.86)+-(512,19.67) 
        };
        \addplot+[error bars,y dir=both, y explicit] coordinates {
            (1,13.36)+-(1,0.05) (2,8.25)+-(2,1.41) (4,5.05)+-(4,0.53) (8,3.49)+-(8,0.45) (16,8.10)+-(16,1.15) (32,25.81)+-(32,4.77) (64,14.01)+-(64,1.61) (128,19.77)+-(128,3.11) (256,34.76)+-(256,10.90) (512,105.07)+-(512,0.66)
        };
        \addplot+[error bars,y dir=both, y explicit] coordinates {
            (1,13.39)+-(1,0.03) (2,8.05)+-(2,1.09) (4,5.13)+-(4,0.64) (8,3.37)+-(8,0.57) (16,7.62)+-(16,1.79) (32,13.01)+-(32,3.05) (64,21.35)+-(64,3.62) (128,21.05)+-(128,7.45) (256,57.98)+-(256,27.06) (512,103.22)+-(512,3.24)
        };
        \addplot+[error bars,y dir=both, y explicit] coordinates {
            (1,13.70)+-(1,0.00) (2,8.08)+-(2,0.39) (4,6.03)+-(4,1.01) (8,2.85)+-(8,0.20) (16,7.92)+-(16,1.01) (32,14.57)+-(32,1.81) (64,19.24)+-(64,2.21) (128,18.49)+-(128,0.48) (256,52.31)+-(256,2.27) (512,111.88)+-(512,4.66) 
        };
        \addplot+[error bars,y dir=both, y explicit] coordinates {
            (1,13.73)+-(1,0.00) (2,13.72)+-(2,0.00) (4,14.49)+-(4,0.02) (8,8.98)+-(8,3.78) (16,15.66)+-(16,2.82) (32,12.59)+-(32,0.01) (64,16.86)+-(64,1.83) (128,21.99)+-(128,1.45) (256,24.41)+-(256,0.44) (512,119.27)+-(512,3.89)
        };
        % \legend {Tiny, Latency-Sequential, Latency-Chunks, Threshold-Sequential, Threshold-Chunks}
        \end{axis}
        \end{tikzpicture}
    }
    \subfloat[Aborts]{
        \label{abortKmeans}
        \begin{tikzpicture}[scale=0.35, baseline]
        \begin{axis}[
            ymode=log,
            width=1.5 \linewidth,
            height=1 \linewidth,
            %media de tempo intruder
            ybar=2.5pt,
            %enlargelimits=0.10,            
            % legend style={at={(0.45,1.1)}, anchor=south, legend columns=-1},
            ylabel=Aborts (log),
            xlabel=Threads,
            symbolic x coords={1, 2, 4, 8, 16, 32, 64, 128, 256, 512},
            xtick=data,
            ymin=0,
            ymax=200000000000,
            bar width=5pt,
            % nodes near coords,
            nodes near coords align={vertical},
        ]
        \addplot+[error bars,y dir=both, y explicit] coordinates {
            (1,0.0)+-(1,0.0) (2,4797421.2)+-(2,479774.7418784778) (4,18101846.8)+-(4,3591708.3539415835) (8,33182746.2)+-(8,6732288.556793311) (16,93057923.8)+-(16,9863144.322012922) (32,171442797.6)+-(32,74829388.10384364) (64,359745893.2)+-(64,213043475.04545623) (128,807753260.2)+-(128,310719932.9857997) (256,26569175006.4)+-(256,4470117158.120106) (512,146296330743.6)+-(512,26886172215.929005) 
        };
        \addplot+[error bars,y dir=both, y explicit] coordinates {
            (1,0.0)+-(1,0.0) (2,144295.6)+-(2,24864.282793597726) (4,579330.6)+-(4,68221.2484981036) (8,1486520.2)+-(8,200173.524986098) (16,16580311.4)+-(16,2227873.6890443857) (32,45063610.6)+-(32,5053643.0020858655) (64,103498202.6)+-(64,18052939.09375) (128,59441333.6)+-(128,8917614.717479235) (256,109146323.8)+-(256,36880499.382352196) (512,294773353.4)+-(512,3281309.7536428105)
        };
        \addplot+[error bars,y dir=both, y explicit] coordinates {
             (1,0.0)+-(1,0.0) (2,141609.1)+-(2,19957.913370139675) (4,570320.1)+-(4,69142.6626511447) (8,1412101.5)+-(8,239771.55806068826) (16,15454899.8)+-(16,3814349.1856658272) (32,39591108.222222224)+-(32,9286948.71968007) (64,81659510.1)+-(64,21584431.489666473) (128,62701912.9)+-(128,22625798.149042867) (256,157892521.1)+-(256,60541141.358650364) (512,267641468.75)+-(512,32356420.707834) 
        };
        \addplot+[error bars,y dir=both, y explicit] coordinates {
            (1,0.0)+-(1,0.0) (2,489474.6)+-(2,95557.56667391652) (4,1982817.6)+-(4,324496.7328282983) (8,3338695.4)+-(8,339537.1836017964) (16,11407592.4)+-(16,1467579.9615602007) (32,38923440.666666664)+-(32,5311247.22872386) (64,42424191.0)+-(64,4907763.0) (128,34826711.0)+-(128,578473.0) (256,128783746.63)+-(256,37482783.86) (512,173847878.28)+-(512,37384778.0)
        };
        \addplot+[error bars,y dir=both, y explicit] coordinates {
            (1,0.0)+-(1,0.0) (2,528574.7)+-(2,29384.0) (4,2157923.0)+-(4,283746.0) (8,3736425.62)+-(8,429388.0) (16,15208687.0)+-(16,6397145.233711557) (32,37279545.0)+-(32,6425946.0) (64,43456253.0)+-(64,3849829.0) (128,49875516.0)+-(128,675193.0) (256,148983127.32)+-(256,31827833.0) (512,187678278.0)+-(512,32738943.73)
        };
        % \legend {Tiny, Latency-Sequential, Latency-Chunks, Threshold-Sequential, Threshold-Chunks}
        \end{axis}
        \end{tikzpicture}
    }
\end{figure}

\end{frame}

\begin{frame} \frametitle{Labyrinth}
    \begin{figure}[!ht]
    \centering
    \includegraphics[scale=0.3]{images/legend}
    
    \subfloat[Tempo de execução]{
        \label{Labyrinth}
        \begin{tikzpicture}[scale=0.35, baseline]
            \begin{axis}[
                width=1.5 \linewidth,
                height=1 \linewidth,
                %media de tempo intruder
                ybar=2.5pt,
                %enlargelimits=0.10,
                % legend style={at={(0.45,1.1)}, anchor=south, legend columns=-1},
                ylabel=Tempo (s),
                xlabel=Threads,
                symbolic x coords={1, 2, 4, 8, 16, 32, 64, 128, 256, 512},
                xtick=data,
                ymin=0,
                ymax=600,
                bar width=5pt,
                % nodes near coords,
                nodes near coords align={vertical},
            ]
            \addplot+[error bars,y dir=both, y explicit] coordinates {
                (1,543.12)+-(1,0.14) (2,307.12)+-(2,3.77) (4,177.18)+-(4,3.79) (8,86.06)+-(8,0.82) (16,50.63)+-(16,0.60) (32,35.89)+-(32,0.53) (64,26.45)+-(64,0.63) (128,25.83)+-(128,0.64) (256,41.98)+-(256,1.85) (512,57.32)+-(512,2.08) 
            };
            \addplot+[error bars,y dir=both, y explicit] coordinates {
                (1,553.57)+-(1,0.26) (2,287.70)+-(2,0.17) (4,165.86)+-(4,0.67) (8,94.13)+-(8,1.50) (16,65.04)+-(16,2.20) (32,37.22)+-(32,1.17) (64,25.60)+-(64,1.00) (128,25.49)+-(128,1.49) (256,26.10)+-(256,1.15) (512,25.80)+-(512,0.75)
            };
            \addplot+[error bars,y dir=both, y explicit] coordinates {
                (1,553.49)+-(1,0.21)(2,287.57)+-(2,0.25)(4,165.74)+-(4,0.99)(8,93.35)+-(8,1.29)(16,65.67)+-(16,6.11)(32,38.25)+-(32,1.76)(64,29.61)+-(64,0.77)(128,26.85)+-(128,2.24)(256,25.63)+-(256,0.42)(512,27.96)+-(512,1.35)
            };
            \addplot+[error bars,y dir=both, y explicit] coordinates {
                (1,553.89)+-(1,0.08) (2,287.30)+-(2,0.32) (4,164.88)+-(4,0.69) (8,93.33)+-(8,0.76) (16,55.06)+-(16,0.46) (32,35.27)+-(32,0.97) (64,26.35)+-(64,1.17) (128,26.33)+-(128,1.74) (256,26.96)+-(256,1.43) (512,29.70)+-(512,2.27) 
            };
            \addplot+[error bars,y dir=both, y explicit] coordinates {
                (1,553.38)+-(1,0.02) (2,287.58)+-(2,0.22) (4,164.99)+-(4,0.38) (8,93.83)+-(8,0.15) (16,55.92)+-(16,0.76) (32,35.17)+-(32,0.71) (64,26.79)+-(64,0.74) (128,25.67)+-(128,0.66) (256,27.99)+-(256,1.60) (512,28.01)+-(512,0.74)
            };
        % \legend {Tiny, Latency-Sequential, Latency-Chunks, Threshold-Sequential, Threshold-Chunks}
        \end{axis}
        \end{tikzpicture}
    }
    \subfloat[Aborts]{
        \label{abortLabyrinth}
        \begin{tikzpicture}[scale=0.35, baseline]
        \begin{axis}[
            ymode=log,
            width=1.5 \linewidth,
            height=1 \linewidth,
            %media de tempo intruder
            ybar=2.5pt,
            %enlargelimits=0.10,
            % legend style={at={(0.45,1.1)}, anchor=south, legend columns=-1},
            ylabel=Aborts (log),
            xlabel=Threads,
            symbolic x coords={1, 2, 4, 8, 16, 32, 64, 128, 256, 512},
            xtick=data,
            ymin=0,
            ymax=4000,
            bar width=5pt,
            % nodes near coords,
            nodes near coords align={vertical},
        ]
        \addplot+[error bars,y dir=both, y explicit] coordinates {
            (1,0.0)+-(1,0.0) (2,12.2)+-(2,2.63) (4,53.2)+-(4,5.94) (8,127.2)+-(8,6.14) (16,258.0)+-(16,24.04) (32,490.2)+-(32,16.41) (64,847.0)+-(64,12.39) (128,1413.4)+-(128,186.85) (256,1959.6)+-(256,67.08) (512,3727.2)+-(512,167.11) 
        };
        \addplot+[error bars,y dir=both, y explicit] coordinates {
            (1,0.0)+-(1,0.0) (2,11.4)+-(2,0.48) (4,56.0)+-(4,4.09) (8,142.0)+-(8,12.85) (16,245.8)+-(16,19.87) (32,491.2)+-(32,25.11) (64,811.8)+-(64,28.18) (128,1334.2)+-(128,63.09) (256,1633.4)+-(256,41.57) (512,2556.6)+-(512,340.72)
        };
        \addplot+[error bars,y dir=both, y explicit] coordinates {
             (1,0.0)+-(1,0.0) (2,12.2)+-(2,1.46) (4,56.4)+-(4,3.61) (8,136.4)+-(8,10.30) (16,245.4)+-(16,19.39) (32,473.2)+-(32,16.47) (64,734.8)+-(64,11.12) (128,1298.4)+-(128,60.09) (256,1677.2)+-(256,35.19) (512,1686.8)+-(512,244.04)
        };
        \addplot+[error bars,y dir=both, y explicit] coordinates {
            (1,0.0)+-(1,0.0) (2,11.8)+-(2,0.97) (4,54.8)+-(4,2.78) (8,134.8)+-(8,4.39) (16,271.2)+-(16,11.35) (32,482.8)+-(32,22.57) (64,822.0)+-(64,36.78) (128,1291.2)+-(128,74.59) (256,1657.8)+-(256,58.13) (512,2571.6)+-(512,174.36)
        };
        \addplot+[error bars,y dir=both, y explicit] coordinates {
            (1,0.0)+-(1,0.0) (2,11.0)+-(2,1.2) (4,55.72)+-(4,10.82) (8,143.2)+-(8,25.75) (16,286.0)+-(16,10.86) (32,496.6)+-(32,19.73) (64,812.0)+-(64,38.57) (128,1238.6)+-(128,53.69) (256,1629.6)+-(256,55.64) (512,2933.4)+-(512,289.46)
        };
        % \legend {Tiny, Latency-Sequential, Latency-Chunks, Threshold-Sequential, Threshold-Chunks}
        \end{axis}
        \end{tikzpicture}
    }
\end{figure}
\end{frame}

\begin{frame} \frametitle{Vacation}
    \begin{figure}[!ht]
    \centering
    \includegraphics[scale=0.3]{images/legend}
    
    \subfloat[Tempo de execução]{
        \label{Vacation}
        \begin{tikzpicture}[scale=0.35, baseline]
        \begin{axis}[
            width=1.5 \linewidth,
            height=1 \linewidth,
            %media de tempo intruder
            ybar=2.5pt,
            %enlargelimits=0.10,
            % legend style={at={(0.5,-0.15)}, anchor=north, legend columns=-1},
            ylabel=Tempo (s),
            xlabel=Threads,
            symbolic x coords={1, 2, 4, 8, 16, 32, 64, 128, 256, 512},
            xtick=data,
            ymin=0,
            ymax=200,
            bar width=5pt,
            % nodes near coords,
            nodes near coords align={vertical},
        ]
        \addplot+[error bars,y dir=both, y explicit] coordinates {
            (1,122.19)+-(1,0.35) (2,184.47)+-(2,3.93) (4,103.92)+-(4,1.04) (8,56.80)+-(8,1.07) (16,31.99)+-(16,0.24) (32,19.69)+-(32,0.52) (64,21.68)+-(64,5.72) (128,23.22)+-(128,0.57) (256,45.58)+-(256,11.99) (512,124.45)+-(512,24.20) 
        };
        \addplot+[error bars,y dir=both, y explicit] coordinates {
            (1,129.87)+-(1,0.52) (2,72.20)+-(2,0.44) (4,39.59)+-(4,0.19) (8,21.14)+-(8,0.09) (16,31.78)+-(16,6.87) (32,21.39)+-(32,4.41) (64,24.02)+-(64,4.93) (128,31.01)+-(128,7.24) (256,23.53)+-(256,3.50) (512,23.21)+-(512,3.03)
        };
        \addplot+[error bars,y dir=both, y explicit] coordinates {
            (1,129.58)+-(1,0.88) (2,71.87)+-(2,0.48) (4,39.90)+-(4,0.39) (8,21.15)+-(8,0.64) (16,26.35)+-(16,4.18) (32,20.69)+-(32,3.15) (64,26.17)+-(64,3.04) (128,38.12)+-(128,12.03) (256,27.50)+-(256,7.32) (512,26.48)+-(512,8.21)
        };
        \addplot+[error bars,y dir=both, y explicit] coordinates {
            (1,128.56)+-(1,0.95) (2,72.45)+-(2,0.41) (4,40.00)+-(4,0.22) (8,21.55)+-(8,0.22) (16,24.76)+-(16,0.20) (32,17.51)+-(32,0.39) (64,19.62)+-(64,2.71) (128,26.34)+-(128,1.81) (256,26.30)+-(256,1.52) (512,25.59)+-(512,1.05) 
        };
        \addplot+[error bars,y dir=both, y explicit] coordinates {
            (1,130.29)+-(1,0.63) (2,72.49)+-(2,0.59) (4,40.83)+-(4,0.31) (8,21.11)+-(8,0.63) (16,26.56)+-(16,0.15) (32,19.76)+-(32,3.23) (64,20.45)+-(64,1.74) (128,29.49)+-(128,3.46) (256,25.76)+-(256,0.95) (512,25.70)+-(512,1.49)
        };
        % \legend {Tiny, Latency-Sequential, Latency-Chunks, Threshold-Sequential, Threshold-Chunks}
        \end{axis}
        \end{tikzpicture}
    }
    \subfloat[Aborts]{
        \label{abortVacation}
        \begin{tikzpicture}[scale=0.35, baseline]
        \begin{axis}[
            ymode=log,
            width=1.5 \linewidth,
            height=1 \linewidth,
            %media de tempo intruder
            ybar=2.5pt,
            %enlargelimits=0.10,
            % legend style={at={(0.5,-0.15)}, anchor=north, legend columns=-1},
            ylabel=Aborts (log),
            xlabel=Threads,
            symbolic x coords={1, 2, 4, 8, 16, 32, 64, 128, 256, 512},
            xtick=data,
            ymin=0,
            ymax=31000000000,
            bar width=5pt,
            % nodes near coords,
            nodes near coords align={vertical},
        ]
        \addplot+[error bars,y dir=both, y explicit] coordinates {
            (1,0.0)+-(1,0.0) (2,69940.4)+-(2,38859.442881235445) (4,123224.4)+-(4,10230.047558051723) (8,421702.0)+-(8,83144.47432752221) (16,858817.0)+-(16,70319.27963794851) (32,5482927.6)+-(32,1822574.306164838) (64,102125071.2)+-(64,75531726.60526115) (128,324906951.0)+-(128,25833176.356745664) (256,5976414236.0)+-(256,1965369780.1742249) (512,25331573456.2)+-(512,5125278647.27645) 
        };
        \addplot+[error bars,y dir=both, y explicit] coordinates {
            (1,0.0)+-(1,0.0) (2,17295.8)+-(2,884.117051074121) (4,58832.4)+-(4,5182.0950049183775) (8,145328.0)+-(8,14502.15570182585) (16,179942.6)+-(16,9939.803108713975) (32,213282.4)+-(32,33756.78063204487) (64,1563534.8)+-(64,303501.08637854987) (128,16550448.4)+-(128,2123035.5644618487) (256,42956098.2)+-(256,6012180.935710481) (512,52269879.0)+-(512,4755551.76397057)
        };
        \addplot+[error bars,y dir=both, y explicit] coordinates {
             (1,0.0)+-(1,0.0) (2,17786.7)+-(2,1249.259304548099) (4,54750.2)+-(4,2738.370420523856) (8,141273.5)+-(8,6123.819286197136) (16,209526.5)+-(16,34736.26421551402) (32,201599.3)+-(32,81127.76465680539) (64,20569626.4)+-(64,12875680.754533986) (128,10897558.0)+-(128,4040389.7657925775) (256,46668204.777777776)+-(256,5450523.901092766) (512,39389491.0)+-(512,16022591.920577014) 
        };
        \addplot+[error bars,y dir=both, y explicit] coordinates {
            (1,0.0)+-(1,0.0) (2,19074.2)+-(2,369.0059078117856) (4,61648.4)+-(4,2902.3763091646124) (8,161693.8)+-(8,9525.331540686655) (16,536547.4)+-(16,18869.180433712536) (32,3237316.8)+-(32,385306.9331768636) (64,38103581.6)+-(64,7477543.014654762) (128,171013189.4)+-(128,21848968.54501459) (256,256035773.6)+-(256,17482704.25370147) (512,263634097.2)+-(512,12981014.911830684)
        };
        \addplot+[error bars,y dir=both, y explicit] coordinates {
            (1,0.0)+-(1,0.0) (2,21093.1)+-(2,312.0) (4,64726.72)+-(4,3041.77) (8,183647.1)+-(8,8673.62) (16,544787.54)+-(16,23782.8) (32,3980021.4)+-(32,552484.0028907986) (64,45836278.2)+-(64,6287845.41) (128,199521897.4)+-(128,10431600.060439752) (256,254781534.6)+-(256,19017136.502195936) (512,266950650.75)+-(512,14874593.300939599)
        };
        % \legend {Tiny, Latency-Sequential, Latency-Chunks, Threshold-Sequential, Threshold-Chunks}
        \end{axis}
        \end{tikzpicture}
    }
\end{figure}

\end{frame}

\begin{frame} \frametitle{Yada}
    \begin{figure}[!ht]
    \centering
    \includegraphics[scale=0.3]{images/legend}

    \subfloat[Tempo de execução]{
        \label{Yada}
        \begin{tikzpicture}[scale=0.35, baseline]
        \begin{axis}[
            width=1.5 \linewidth,
            height=1 \linewidth,
            %media de tempo intruder
            ybar=2.5pt,
            %enlargelimits=0.10,
            % legend style={at={(0.45,1.1)}, anchor=south, legend columns=-1},
            ylabel=Tempo (s),
            xlabel=Threads,
            symbolic x coords={1, 2, 4, 8, 16, 32, 64, 128, 256, 512},
            xtick=data,
            ymin=0,
            ymax=200,
            bar width=5pt,
            % nodes near coords,
            nodes near coords align={vertical},
        ]
        \addplot+[error bars,y dir=both, y explicit] coordinates {
            (1,9.95)+-(1,0.09) (2,19.42)+-(2,0.50) (4,15.05)+-(4,1.07) (8,12.75)+-(8,1.30) (16,8.85)+-(16,1.90) (32,15.25)+-(32,2.69) (64,18.85)+-(64,2.74) (128,51.34)+-(128,5.50) (256,89.66)+-(256,12.90) (512,189.43)+-(512,7.63) 
        };
        \addplot+[error bars,y dir=both, y explicit] coordinates {
            (1,14.09)+-(1,0.09) (2,9.88)+-(2,0.04) (4,7.22)+-(4,0.04) (8,4.78)+-(8,0.29) (16,4.19)+-(16,0.35) (32,4.74)+-(32,0.50) (64,5.38)+-(64,1.21) (128,12.38)+-(128,3.72) (256,13.14)+-(256,2.01) (512,15.43)+-(512,3.54)
        };
        \addplot+[error bars,y dir=both, y explicit] coordinates {
            (1,14.31)+-(1,0.16) (2,9.98)+-(2,0.07) (4,7.23)+-(4,0.11) (8,4.19)+-(8,1.93) (16,3.85)+-(16,0.97) (32,3.96)+-(32,1.37) (64,7.11)+-(64,1.83) (128,11.47)+-(128,2.48) (256,11.12)+-(256,5.94) (512,14.17)+-(512,1.47)
        };
        \addplot+[error bars,y dir=both, y explicit] coordinates {
            (1,14.17)+-(1,0.05) (2,14.17)+-(2,0.11) (4,14.20)+-(4,0.08) (8,11.61)+-(8,2.13) (16,13.53)+-(16,1.93) (32,15.15)+-(32,4.48) (64,14.19)+-(64,2.00) (128,15.69)+-(128,2.21) (256,27.92)+-(256,0.20) (512,22.18)+-(512,1.28) 
        };
        \addplot+[error bars,y dir=both, y explicit] coordinates {
            (1,14.35)+-(1,0.05) (2,14.29)+-(2,0.04) (4,14.31)+-(4,0.08) (8,14.31)+-(8,0.05) (16,15.07)+-(16,4.06) (32,13.64)+-(32,1.83) (64,6.83)+-(64,1.09) (128,13.37)+-(128,2.19) (256,15.65)+-(256,0.16) (512,15.73)+-(512,0.14)
        };
        % \legend {Tiny, Latency-Sequential, Latency-Chunks, Threshold-Sequential, Threshold-Chunks}
        \end{axis}
        \end{tikzpicture}
    }
    \subfloat[Aborts]{
        \label{abortYada}
        \begin{tikzpicture}[scale=0.35, baseline]
        \begin{axis}[
            ymode=log,
            width=1.5 \linewidth,
            height=1 \linewidth,
            %media de tempo intruder
            ybar=2.5pt,
            %enlargelimits=0.10,
            % legend style={at={(0.45,1.1)}, anchor=south, legend columns=-1},
            ylabel=Aborts (log),
            xlabel=Threads,
            symbolic x coords={1, 2, 4, 8, 16, 32, 64, 128, 256, 512},
            xtick=data,
            ymin=0,
            ymax=400000000000,
            bar width=5pt,
            % nodes near coords,
            nodes near coords align={vertical},
        ]
        \addplot+[error bars,y dir=both, y explicit] coordinates {
            (1,0.0)+-(1,0.0) (2,53805739.2)+-(2,945974.4856792702) (4,136025541.2)+-(4,4837342.13) (8,281305083.6)+-(8,28618515.67361887) (16,455944205.2)+-(16,138166587.07) (32,1205121875.2)+-(32,315403375.80832976) (64,4297295036.4)+-(64,1609418223.03) (128,27371816768.6)+-(128,18399453268.002537) (256,230017372308.6)+-(256,19982225506.96) (512,156775482845.2)+-(512,5418181512.20) 
        };
        \addplot+[error bars,y dir=both, y explicit] coordinates {
            (1,0.0)+-(1,0.0) (2,1580633.2)+-(2,8478.500749542929) (4,3800972.6)+-(4,20650.110223434644) (8,6149664.4)+-(8,75882.93103880476) (16,7207458.2)+-(16,106075.91092684522) (32,7097792.4)+-(32,149914.3555148739) (64,7245601.6)+-(64,297306.3815181907) (128,6626592.6)+-(128,1601916.1098185633) (256,11035923.666666666)+-(256,1354186.7993902795) (512,11463741.25)+-(512,1422857.6912357355)
        };
        \addplot+[error bars,y dir=both, y explicit] coordinates {
             (1,0.0)+-(1,0.0) (2,1586885.6)+-(2,17013.20809959133) (4,3864768.25)+-(4,23733.99245149244) (8,6283745.6)+-(8,79485.5) (16,7192878.4)+-(16,178424.65) (32,7378729.9)+-(32,18573.11) (64,4723078.0)+-(64,3928.0) (128,6736182.0)+-(128,37123.3) (256,8828877.0)+-(256,12847.0) (512,11637482.0)+-(512,82034.0) 
        };
        \addplot+[error bars,y dir=both, y explicit] coordinates {
            (1,0.0)+-(1,0.0) (2,1502389.8)+-(2,8336.5398735266599) (4,34570.6)+-(4,17081.853735470282) (8,4673390.0)+-(8,3789634.775837482) (16,2964970.8)+-(16,290439.4272928145) (32,12301780.8)+-(32,1054972.452489657) (64,9002166.4)+-(64,735560.767845634) (128,14938022.8)+-(128,3727246.128412794) (256,8642330.0)+-(256,5465830.0) (512,11928378.731)+-(512,74928.0)
        };
        \addplot+[error bars,y dir=both, y explicit] coordinates {
            (1,0.0)+-(1,0.0) (2,1483981.3)+-(2,5287.71) (4,34983.38)+-(4,21305.7) (8,4728374.12)+-(8,2263748.1) (16,5260517.0)+-(16,3862894.25) (32,9462941.0)+-(32,3192100.38) (64,8837692.27)+-(64,384722.0) (128,13729279.5)+-(128,110344.5) (256,8874348.23)+-(256,63743.0) (512,11973628.0)+-(512,837874.02)
        };
        % \legend {Tiny, Latency-Sequential, Latency-Chunks, Threshold-Sequential, Threshold-Chunks}
        \end{axis}
        \end{tikzpicture}
    }
\end{figure}

\end{frame}

\section{Conclusão}
\begin{frame} \frametitle{Conclusão}
    \begin{block}{Concluão}
        \begin{itemize}
        	\item Apresentamos o Escalonador LTMS;
        	\item Foi prototipado utilizando a TinySTM; e
        	\item Possui 3 etapas de execução.
        	% \item Heurísticas de distribuição; e
        	% \item Heurísticas de migração.
        \end{itemize}
    \end{block}
\end{frame}

\begin{frame} \frametitle{Conclusão}
    \begin{block}{Analise}
        \begin{itemize}
        	\item Aplicações com conjunto pequeno de leitura e escrita;
        	\item Tamanho médio de transação apresentou melhor execução;
        	\item Alta contenção apresentou melhor tempo execução;
        	\item Melhor caso com redução de 96\% no tempo de execução;
        	\item Melhor caso com redução de 99\% na ocorrência de aborts; e
        	\item Latency apresentou resultados melhores para maioria dos testes.
        \end{itemize}
    \end{block}
\end{frame}

\begin{frame} \frametitle{Conclusão}
    \begin{block}{Trabalhos futuros}
        \begin{itemize}
        	\item Novas Heurísticas de distribuição;
        	\item Juntar as heurísticas de migração Threshold e Latency; e
        	\item Impacto energético dos escalonadores de STM.
        \end{itemize}
    \end{block}
\end{frame}

\maketitle
\end{document}

